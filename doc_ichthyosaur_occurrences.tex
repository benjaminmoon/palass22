%!TEX program = lualatex

\documentclass[british]{article}

\title{Ichthyosauromorpha Occurrences}
\author{Benjamin Moon}
\date{\printdate{2020-03-10}\sidenote{Version~\version}}

\usepackage[osf, proportional]{sourceserifpro}
\usepackage[osf, proportional]{sourcesanspro}
\usepackage[osf]{sourcecodepro}

\usepackage{tufte-lualatex}

\setVersion{0.1}

\addbibresource{zotero.bib}

\begin{document}

\maketitle

\tableofcontents

\section{Occurrence Data}%
\label{sec:occurrence-data}

Initial occurrence data were downloaded from the Palaeobiology Database on
\printdate{2020-03-10}. Entries were vetted for consistency and occurrence dates
were re-evaluated to as precise intervals as possible. Absolute ages were  taken
from \textcite{Ogg2016} to ammonite/conodont biozone level where possible using
the TSCreator tool (\url{https://timescalecreator.org/index/index.php}).

\subsection{Acamptonectes densus}%
\label{sub:acamptonectes-densus}

Three specimens of \emph{Acamptonectes densus} are recorded by
\textcite{Fischer2012PO} from the Hauterivian. The two British specimens are
from the \textallsc{D2D} and \textallsc{D2C}, which are identified to Lower
Neocomian. Bed \textallsc{D2D} is identified as the most basal
\emph{Endomoceras} bed,\autocite{Hopson2008} which approximately matches Boreal
Realm ammonite zones from the base of the Hauterivian at
\SI{134.69}{\mega\annum}.\autocite{Ogg2016ACGTSa} \emph{Endomoceras} ammonite
biozones correspond to the first two Tethyan ammonite biozones
(\emph{Acanthodiscus radiatus} and \emph{Crioceratites loryi}).

A specimen referred to \emph{Acaptonectes densus} from Cremlingen, Germany is
identified to \emph{Simbirskites (Milankowskia) concinnus/staffi} Ammonite
Biozone.\autocite{Siebertz2008BNS} This is equavalent to the \emph{Milankowska
speetonensis} Ammonite Biozone in the Boreal Realm.\autocite{Ogg2016ACGTSa}

\begin{table}[htb]
    \footnotesize
    \sidecaption[][-3em]{Occurrence ages of \emph{Acamptonectes densus.}}
    \begin{tabu}{>{\em}lXSS}
        \toprule
        \emph{Taxon}         & Stratigraphy                                            & {Max Age (Ma)} & {Min Age (Ma)} \\
        \midrule
        Acamptonectes densus & \emph{Endomoceras} ammonite biozones, Early Hauterivian & 134.71         & 133.87         \\
        Acamptonectes densus & \emph{Milankowskia speetonensis} Ammonite Biozone       & 133.32         & 132.85         \\
        \bottomrule
    \end{tabu}
\end{table}


\subsection{Acuetzpalin carranzai}%
\label{sub:acuetzpalin-carranzai}

\emph{Acuetzpalin carranzai} is found in the La Casita Formation of nothern
Maxeico, but is limited only to the Kimmeridgian
Stage.\autocite{BarrientosLara2020JSAES} Further marine vertebrate remains are
found through this formation, and others coeval formations, and certain
concentrations have been correlated to \emph{Hybonoticeras beckeri} Ammonite
Biozone (\SIrange{153.55}{152.06}{\mega\annum}).\autocite{Zell2014JSAES}

\begin{table}[htb]
    \footnotesize
    \sidecaption[][-3em]{Occurrence ages of \emph{Acuetzpalin carranzai}.}
    \begin{tabu}{>{\em}lXSS}
        \toprule
        Taxon & Stratigraphy & {Max Age (Ma)} & {Min Age (Ma)} \\
        \midrule
        Acuetzpalin carranzai & Kimmeridgian & 157.25 & 152.06 \\
        \bottomrule
    \end{tabu}
\end{table}

\subsection{Aegirosaurus leptospondylus}%
\label{sub:aegirosaurus-leptospondylus}

The neotype and one referred specimen of \emph{Aegirosaurus leptospondylus} are
referred to Malm ζ2b, the upper part of the part of the Altmühtal
Formation.\autocite{Bardet2000JP} These lithographic limestones are in the
\emph{Hybonoticeras hybonotum} Ammonite Biozone, the most basal of the
Tithonian. Together with the other two specimens referred by
\textcite{Bardet2000JP} all known occurrences of \emph{Aegirosaurus
leptospondylus} are from the same general time and locality.

\begin{table}[htb]
    \footnotesize
    \sidecaption[][-3em]{Occurrence ages of \emph{Aegirosaurus leptospondylus}.}
    \begin{tabu}{>{\em}lXSS}
        \toprule
        Taxon                       & Stratigraphy & {Max Age (Ma)} & {Min Age (Ma)} \\
        \midrule
        Aegirosaurus leptospondylus & \emph{Hybonoticeras hybonotum} Ammonite Biozone & 152.06 & 150.94 \\
        \bottomrule
    \end{tabu}
\end{table}

\subsection{Arthropterygius}%
\label{sub:arthropterygius}

Holotype material of \emph{Ophthalmosaurus chrisorum} was referred to the new
genus \emph{Arthropterygius}.\autocite{Russell1993BGSC,Maxwell2010JVP} These
original finds are from the Ringnes Formation of Melville Island, Canada, but
aren’t more certainly dated than Oxfordian–Kimmeridgian. Numerous additional
remains have since been referred to \emph{Arthropterygius}, which now contains
several species. \emph{Arthropterygius thalassonotus} is from the Tithonian Vaca
Muerta Formation of Neuquén Province, Argentina. This is referred to “late
Tithonian”\autocite[5]{Campos2019ZJLS}, and this is taken as a true referral
age, although itm ay be an informal reference.

A revision of Russian and Spitsbergen material has referred several general
(\emph{Palvennia, Janusaurus, Keilhauia}) to \emph{Arthropterygius} and added
more specimens to \emph{Arthropterygius chrisorum}, also extending its
occurrences to the Berriasian.\autocite{Zverkov2019P}

Material from Svalbard is dated using the Boreal ammonites correlate to Siberia
and to Russia. \emph{Arthropterygius} cf. \emph{chrisorum} and
\emph{Undorosaurus gorodischensis} (formerly \emph{Cryopterygius kristiansenae})
both have specimens from \emph{Crendonites anguinus}  Ammonite
Biozone,\autocite{Zverkov2019JSP,Zverkov2019P} which is
here correlated with the \emph{Taimyrosphinctes excentricus} Ammonite Biozone.
\emph{Garniericeras catenulatum} Ammonite Biozone is treated as equivalent to
\emph{Craspedites subdites} Ammonite Biozone.

\begin{table}[htb]
    \footnotesize
    \sidecaption[][-3em]{Occurrence ages of \emph{Arthropterygius}.}
    \begin{tabu}{>{\em}lXSS}
        \toprule
        Taxon                 & Stratigraphy & {Max Age (Ma)} & {Min Age (Ma)} \\
        \midrule
        Arthropterygius chrisorum & Oxfordian–Kimmeridgian & 161.18 & 152.06 \\
        Arthropterygius chrisorum & Lower Berriasian & 145.73 & 143.57 \\
        Arthropterygius chrisorum & Lower Berriasian & 145.73 & 143.57 \\
        Arthropterygius chrisorum & \emph{Dorsoplanites panderi} Ammonite Biozone & 149.59 & 147.93 \\
        Arthropterygius lundi & \emph{Dorsoplanites panderi} Ammonite Biozone & 149.59 & 147.93 \\
        Arthropterygius thalassonotus & Upper Tithonian & 147.72 & 145.73 \\
        Arthropterygius volgensis & \emph{Dorsoplanites panderi} Ammonite Biozone & 149.59 & 147.93 \\
        \bottomrule
    \end{tabu}
\end{table}


\subsection{Ophthalmosaurus}%
\label{sub:ophthalmosaurus}

Previously referred to \emph{Baptanodon}, it’s currently found most often to be
a species within \emph{Ophthalmosaurus}, although the exact affinities to
\emph{Ophthalmosaurus icenicus} are uncertain. Several specimens of Jurassic
ichthyosaurs have been found in the American mid-West, primarily from the
Sundance Formation. More recent finds are most often found in the Redwater Shale
member, however historical remains don’t have such precise lithostratigraphy.

The Redwater Shale Formation is divide into lower and upper sections, ranging
from Late Callovian–Middle Oxfordian.\autocite{Kvale2001P} The lower portion
contains ammonites referred to the genus \emph{Quenstedtoceras} so is here
referred to Late Callovian–Early Oxfordian, covering the two
\emph{Quenstedoceras} ammonite biozones in the Tethyan Realm. The upper part
covers \emph{Cardioceras cordatum}–\emph{Perisphinctes plicatum} ammonite
biozones. The \emph{Baptanodon} beds of Gilmore and Shirley Stage of Knight are
synonymous with the Redwater Shale Member of the Sundance Formation.

Other synonymous taxa here assigned to \emph{Ophthalmosaurus natans} include
\emph{Apatodonosaurus grayi} and \emph{Baptanodon}, both from
Wyoming.\autocite{Mehl1928JSLDUG,Massare2014GM,Moon2016MPS}

\begin{table}[htb]
    \footnotesize
    \sidecaption[][-3em]{Occurrence ages of \emph{Ophthalmosaurus natans}.}
    {\addfontfeatures{Numbers = Tabular}
    \begin{tabu}{>{\em}lXSS}
        \toprule
        Taxon                  & Stratigraphy & {Max Age (Ma)} & {Min Age (Ma)} \\
        \midrule
        Ophthalmosaurus natans & Oxfordian & 163.1 & 157.25 \\
        Ophthalmosaurus natans & \emph{Cardiocera cordatum–Perisphinctes plicatilis} ammonite biozones & 161.18 & 159.88 \\
        \bottomrule
    \end{tabu}}
\end{table}

\subsection{Undorosaurus}%
\label{sub:undorosaurus}



\printbibliography
    
\end{document}
